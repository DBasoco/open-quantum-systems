\documentclass[12pt]{article}

\usepackage{mathtools}
\usepackage{blindtext}

\usepackage[margin=1in]{geometry}

\newcommand*{\eu}{e}
\newcommand*{\iu}{i}

\DeclarePairedDelimiter{\bra}{\langle}{\rvert}
\DeclarePairedDelimiter{\ket}{\lvert}{\rangle}
\DeclarePairedDelimiterX{\braket}[2]{\langle}{\rangle}
  {#1\,\delimsize\vert\,\mathopen{}#2}

\usepackage{biblatex}
\addbibresource{report.bib}

\usepackage[hidelinks]{hyperref}

\title{Solving Decoherence Channels on Open Quantum Systems}
\author{David Basoco \and Jack Hetherington \and Davis Rash \and Tim Ross}
\date{November 8, 2023}

\begin{document}
  \maketitle

  \section{Introduction}
  No quantum system is perfectly isolated from the environment. The case for the closed quantum systems, where the equation of motion is fully describing the state dynamics is the Schr\"{o}dinger equation. Generally, the dynamics of open quantum systems are described by a master equation, usually Lindbladian. Only under certain assumptions, known as the Born-Markov approximation, one can get an equation in this form able to describe the physical evolution of the quantum states. When these assumptions are not satisfied, we enter the non-Markovian realm. This distinction and breakdown of equations in changing scope illustrates how the open quantum systems theories are far from being completely solved.  

  We will cover several test beds for experimental research into open quantum systems. These test bed classs include Depolarizing and Puali channels, Markovian Reservoir Engineering and Amplitude damping. These experimental results demonstrate the versatile nature of quantum simulators and how they can be used for verifying and exploration into open quantum physics. 
  \blindtext

  \section{Applications}
  As we continue to make quantum computers larger, we generate macro systems that begin to enter the classical realm. As these systems grow the natural decoherence of quantum phenomenon rapidly increases. This results in qunatum systems that rapidly decohere in the presence of the environment or other micro states. The range is vast for this field of inquiry. From solid state physics to quantum field theory, from quantum chemistry and biology to quantum thermodynamics, numerous fields stand to benefit from understanding how macro quantum states interact and form the classical regime we interact with daily.
  \blindtext

  \section{Depolarizing and Pauli Channels}
  \blindtext

  \section{Markovian Reservoir Engineering}
  \blindtext

  \section{Amplitude Damping}
  The circuit to implement the amplitude damping channel with the non-Markovianity witness as shown below. The witness is based on the behaviour of an initial maximally entangled state of qubit and auxiliary ancilla, and requires only the measurement of the expectation values of local observables

  \begin{figure}[h]
    \centering
    \includegraphics[width=0.7\textwidth]{images/amplitude_damping_yy_circuit}
    \caption{The circuit to implementing amplitude damping with the non-Markovianity witness %
            \label{fig:amplitude_damping_circuit}}
  \end{figure}
  For an arbitrary pure state of the system $\ket{\psi}_{s} = \alpha\ket{0}_{s} + \beta\ket{1}_{s}$, and setting the state of the environment to vacuum $\ket{0}_{e}$, the two gates between the system and environment qubits transform the join state into 
  \begin{equation}
    \alpha\ket{0}_{1}\ket{0}_{2} + \beta\cos \theta \ket{1}_{1}\ket{0}_{2} + \beta\sin \theta \ket{0}_{1}\ket{1}_{2}
  \end{equation}

  Where $q_{1}$ and $q_{2}$ are the system and environment qubits respectfully. Therefore, identifying the states $\ket{0}_{1}$ and $\ket{1}_{1}$ with the ground state and excited states respectfully, and by choosing $\theta = \arccos c_{1}(t)$, we get a reduced state of the system such that

  \begin{equation}
    \rho_{s}(t) = \begin{pmatrix}
        \lvert c_{1}(t)\rvert^2 & c_{0}^{*}c_{1}(t) \\
        c_{0}c_{1}^{*}(t) & 1 - \lvert c_{1}(t)\rvert^2 
    \end{pmatrix}
  \end{equation}

  These factors $c_{0}$ and $c_{1}$ are time dependent factors of the wave equation. With a decay rate $\gamma(t)$ form
  \begin{equation}
    \gamma(t) = -2\mathcal{R} \left[ \frac{\dot{c}_{1}(t)}{c_{1}(t)} \right],
  \end{equation}
  with 
  \begin{equation}
    c_{1}(t) = c_{1}(0)e^{\frac{-\lambda t}{2}} \left[ \cosh (\frac{\lambda t}{2} \sqrt{1 - 2R}) + \frac{1}{\sqrt{1 - 2R}} \sinh (\frac{\lambda t}{2} \sqrt{1 - 2R}) \right]
  \end{equation}
  This R value is crucial at it is the ratio between the coupling strength and the width of the spectrum. This coefficient will be varied to examine the behavior of the damping. By measuring the witness across the various R values we can see the presence of memory effects with the environment.


\section{Trace Distance}
        The goal for the trace distance, is to determine the difference between the expected result and the simulated noise result. We can use that the trace distance is half of the trace of the absolute value of the difference of the two density matrices. \\
        Qiskit allows you to obtain the density matrix for a given quantum circuit fairly easily. However it is impossible to obtain the density matrix for a measured set of results. But, we can obtain an approximate density matrix by calculating the probability of being in a specified state then creating an approximate wave function, which we can then use to create an approximate density matrix. From here we can calculate the trace distance.  

  \section{Results}
  \subsection{Trace Distance}
        We were unable to measure the trace distance for our given set of cases, for the reservoir engineering it was not possible to create a density matrix from the results that matched in dimension of the one generated by the circuit without affecting the results. (Add in stuff on other forms?)

  \begin{figure}[h]
    \centering
    \includegraphics[width=\textwidth]{images/reservoir-engineering-simulation}
    \caption{Simulation of a quantum circuit.%
      \label{fig:reservoir-engineering-simulation}}
  \end{figure}
  \subsection{Amplitude Damping}
  We plot the dynamics of the entanglement witness for the amplitude damping channel.
  
  \begin{figure}[h]
    \centering
    \includegraphics[width=0.7\textwidth]{images/amplitude_damping_population_non_markovianity}
    \caption{The comparison for damping time at 4 different R values %
            \label{fig:amplitude_damping_population}}
  \end{figure}

  The witness clearly shows oscillatory behaviour, and therefore properly signals the presence of memory effects.


  \printbibliography
\end{document}